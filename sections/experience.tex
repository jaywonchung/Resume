%-------------------------------------------------------------------------------
%	SECTION TITLE
%-------------------------------------------------------------------------------
\cvsection{Experiences}


%-------------------------------------------------------------------------------
%	CONTENT
%-------------------------------------------------------------------------------
\begin{cventries}

%---------------------------------------------------------
  \cventryexp
    {Advisor: Mosharaf Chowdhury}
    {Energy-Efficient Systems for Machine Learning}
    {SymbioticLab, UMich}
    {Sep 2021 - Present}
    {
      \begin{cvitems}
        \item {\href{https://ml.energy/zeus}{\myuline{\textit{Zeus}}}: Discovered the trade-off between DNN training time and energy. Designed a Multi-Armed Bandit solution for time-energy optimization.}
        \item {\href{https://ml.energy/zeus/research\_overview/perseus}{\myuline{\textit{Perseus}}}: A system for energy-efficient large model training (e.g., LLM). Cuts up to 30\% energy without slowdown. Open-sourced as part of Zeus.}
        \item {\href{https://ml.energy/leaderboard}{\myuline{\textit{ML.ENERGY Leaderboard \& Colosseum}}}: The first systematic benchmark and interactive comparison service for GenAI energy consumption, including LLMs, Multimodal LLMs, and Diffusion models.}
      \end{cvitems}
    }
    
%---------------------------------------------------------
  \cventryexp
    {Manager: Mengchi Zhang}
    {MoE training support on MTIA platforms}
    {AI and Systems Co-Design Team, Meta}
    {May 2025 - Aug 2025}
    {
      \begin{cvitems}
        \item {MoE (Mixture-of-Experts) training support on MTIA platforms. Fixed issues and closed gaps across the entire stack, including MTIA kernels, PyTorch MTIA backend, collective communication, and Meta's internal large model training framework.}
      \end{cvitems}
    }
    
%---------------------------------------------------------
  \cventryexp
    {Advisor: Byung-Gon Chun}
    {Software Systems for Machine Learning}
    {Software Platform Lab, SNU}
    {Mar 2020 - Jun 2021}
    {
      \begin{cvitems}
        \item {\textit{Crane}: A GPU cluster manager for AutoML workloads. Built a Kubernetes backend that scaled to 288 GPUs. Contributed core features such as automatic bootstrapping on Docker Swarm and Kubernetes and log streaming through the EFK (Elasticsearch - Fluent Bit - Kibana) stack.}
      \end{cvitems}
    }
    
%---------------------------------------------------------
  \cventryexp
    {Advisor: Soo-Mook Moon}
    {Online Model Specialization for Edge Video DNN Inference}
    {Virtual Machine and Optimization Lab, SNU}
    {Dec 2019 - Jun 2020}
    {
      \begin{cvitems}
      \item {\href{https://github.com/jaywonchung/shadowtutor}{\myuline{\textit{ShadowTutor}}}: Knowledge distillation from the server to the edge device reduced network data transfer by 95\% and increased throughput by 3x.}
      \end{cvitems}
    }
    
%---------------------------------------------------------
  \cventryexp
    {Advisor: Kyoung Mu Lee}
    {Few-Shot Learning with Meta-Learning}
    {Computer Vision Lab, SNU}
    {Jun 2019 - Dec 2019}
    {
      \begin{cvitems}
        \item {Designed improved meta-initialization methods for Model-Agnostic Meta-Learning (MAML) with neural memory modules and convex programs.}
      \end{cvitems}
    }
    
%---------------------------------------------------------
  \cventryexp
    {Advisor: Jongho Lee}
    {Quantitative Susceptibility Mapping (QSM) with Deep Learning}
    {Lab of Imaging Science and Technology, SNU}
    {Jun 2019 - Aug 2019}
    {
      \begin{cvitems}
      \item {Created a deep learning pipeline for QSM, a 3D MRI medical imaging task, including preprocessing, augmentation, and modeling (\href{https://github.com/jaywonchung/CAD-QSMNet}{\myuline{CAD-QSMNet}}).}
      \end{cvitems}
    }
%---------------------------------------------------------

\end{cventries}
